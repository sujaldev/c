%! Author = Sujal Singh
%! Date = 1/17/25

% Preamble
%! suppress = FileNotFound
\documentclass[11pt]{ipu-ai}
\doctitle{OOPS With Java Lab (ARI256)}

% Packages
\usepackage{amsmath}
\usepackage{amssymb}
\usepackage[T1]{fontenc}
\usepackage[
    pdftitle={OOPS With Java Lab Practical File},
    pdfsubject={OOPS With Java Lab Practical File},
    pdfauthor={Sujal Singh},
    pdfdisplaydoctitle,
    hidelinks,
]{hyperref}
\usepackage{tabularray}
\usepackage{tikz}
\usepackage[indLines]{algpseudocodex}
\usepackage{enumitem}


\renewcommand{\experiment}[2]{%
%    \newpage%
    \begin{center}%
        \textbf{\Huge Experiment--#1}\\[30pt]%
    \end{center}%
    \begin{tabularsection}{Aim}%
        ~\\#2\\%
    \end{tabularsection}}

% Document
\begin{document}
    \maketitle
    \pagenumbering{gobble}
\pagestyle{empty}
\begin{center}
    \textbf{\huge Index} \\[20pt]
    \begin{tblr}{rows={50pt},colspec={|Q[m,c]|Q[12,m]|Q[3,c]|},hlines,vlines,
        cell{2-Z}{1} = {cmd=\textbf{\the\numexpr\arabic{rownum}-1}.},
        cell{1}{2}={c}}
        \textbf{No.} & \textbf{Question} & \textbf{Remarks} \\
        &%
        Water Jug Problem
        & \\
        &%
        8 -- Tile Problem
        & \\
        &%
        Breadth First Search
        & \\
        &%
        Depth First Search
        & \\
        &%
        Graph coloring problem
        & \\
        &%
        AO* Search
        & \\
        &%
        Performing operations on array using NumPy.
        & \\
        &%
        Using pandas in python, create data frames using the same.
        & \\
    \end{tblr}
\end{center}
    % ---------------------------------------------------------------------------------------------------------------- %

    %------------------------------------------------------------------------------------------------------------------%
    %------------------------------------------------------------------------------------------------------------------%
    \experiment{1}{Writing a program, compiling and running using command line in Java.}\\%
    \begin{code}
        {Program}{java}
public class q1 {
    public static void main(String[] args) {
        System.out.println("Hello, World!");
    }
}
    \end{code}%
    \begin{code}
        {Output}{text}
Hello, World!
    \end{code}
    \vspace*{10pt}

    % ---------------------------------------------------------------------------------------------------------------- %
    % ---------------------------------------------------------------------------------------------------------------- %

    \experiment{2(a)}{Write a program to show if the given number is a prime number using for loop.}%
    \begin{code}
        {Program}{java}
public class q2a {
    public static void main(String[] args) {
        int n = 67;

        for (int i = 2; i <= (n / 2); i++) {
            if (n % i == 0) {
                System.out.println("Composite number.");
                return;
            }
        }
        System.out.println("Prime number.");
    }
}
    \end{code}%
    \begin{code}
        {Output}{text}
Prime number.
    \end{code}
    \newpage%

    % ---------------------------------------------------------------------------------------------------------------- %
    % ---------------------------------------------------------------------------------------------------------------- %

    \experiment{2(b)}{Write a program to show if the given number is a prime number using while loop.}%
    \begin{code}
        {Program}{java}
public class q2b {
    public static void main(String[] args) {
        int n = 67;

        int i = 2;
        while (i <= (n / 2)) {
            if (n % i == 0) {
                System.out.println("Composite number.");
                return;
            }
            i += 1;
        }
        System.out.println("Prime number.");
    }
}
    \end{code}%
    \begin{code}
        {Output}{text}
Prime number.
    \end{code}

    % ---------------------------------------------------------------------------------------------------------------- %

\end{document}