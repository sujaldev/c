\startassignment{Introduction to R Software.}

%----------------------------------------------------------------------------------------------------------------------%
\firstquestion{Perform operations like assignment, addition and multiplication on variables.}
\begin{code}
    {Program}{r}
a <- 42
b <- 24
a + b
a * c
\end{code}
\begin{code}
    {Output}{text}
[1] 66
[1] 1008
\end{code}
%----------------------------------------------------------------------------------------------------------------------%

%----------------------------------------------------------------------------------------------------------------------%
\question{Make a vector and assign values to it and access 4\textsuperscript{th}, 3\textsuperscript{rd} and
6\textsuperscript{th} element and access all values from 3\textsuperscript{rd} element upto the length of vector.}
\begin{code}
    {Program}{r}
myvector <- c(0, 10, 20, 30, 40 , 50, 60, 70, 80)

myvector[4]
myvector[3]
myvector[6]
myvector[0:length(myvector)]
\end{code}
\begin{code}
    {Output}{text}
[1] 30
[1] 20
[1] 50
[1]  0 10 20 30 40 50 60 70 80
\end{code}
%----------------------------------------------------------------------------------------------------------------------%

%----------------------------------------------------------------------------------------------------------------------%
\question{Perform arithmetic operations on two vectors of different length.}
\begin{code}
    {Program}{r}
a <- c(0, 1, 1, 2, 3, 5, 8, 13, 21)
b <- c(21, 34, 55)

a + b
a * b
\end{code}
\begin{code}
    {Output}{text}
[1] 21 35 56 23 37 60 29 47 76
[1]    0   34   55   42  102  275  168  442 1155
\end{code}
\newpage
%----------------------------------------------------------------------------------------------------------------------%

%----------------------------------------------------------------------------------------------------------------------%
\question{Create a sequence of numbers from 1 to 20 with a difference of 2 between each number using the \texttt{seq()}
function.}
\begin{code}
    {Program}{r}
seq(from = 1, to = 20, by = 2)
\end{code}
\begin{code}
    {Output}{text}
[1]  1  3  5  7  9 11 13 15 17 19
\end{code}
%----------------------------------------------------------------------------------------------------------------------%

%----------------------------------------------------------------------------------------------------------------------%
\question{Create a matrix and access it's elements.}
\begin{code}
    {Program}{r}
source <- c(
  1, 2, 3, 4,
  5, 6, 7, 8,
  9, 10, 11, 1
)
M <- matrix(source, nrow = 3, ncol = 4)
M
M[2,3]
\end{code}
\begin{code}
    {Output}{text}
     [,1] [,2] [,3] [,4]
[1,]    1    4    7   10
[2,]    2    5    8   11
[3,]    3    6    9    1
[1] 8
\end{code}
%----------------------------------------------------------------------------------------------------------------------%

%----------------------------------------------------------------------------------------------------------------------%
\question{Create two matrices using vectors and perform element multiplication, matrix operations on matrices.}
\begin{code}
    {Program}{r}
a <- matrix(seq(from = 1, to = 25), nrow = 5, ncol = 5)
b <- matrix(seq(from = 26, to = 50), nrow = 5, ncol = 5)
a * b
\end{code}
\begin{code}
    {Output}{text}
     [,1] [,2] [,3] [,4] [,5]
[1,]   26  186  396  656  966
[2,]   54  224  444  714 1034
[3,]   84  264  494  774 1104
[4,]  116  306  546  836 1176
[5,]  150  350  600  900 1250
\end{code}
\newpage
\pagestyle{fancy}
%----------------------------------------------------------------------------------------------------------------------%

%----------------------------------------------------------------------------------------------------------------------%
\question{Using the two matrices perform transpose operations on matrices.}
\begin{code}
    {Program}{r}
a <- matrix(seq(from = 1, to = 20), nrow = 4, ncol = 5)
b <- matrix(seq(from = 21, to = 40), nrow = 4, ncol = 5)
t(a)
t(b)
\end{code}
\begin{code}
    {Output}{text}
     [,1] [,2] [,3] [,4]
[1,]    1    2    3    4
[2,]    5    6    7    8
[3,]    9   10   11   12
[4,]   13   14   15   16
[5,]   17   18   19   20
     [,1] [,2] [,3] [,4]
[1,]   21   22   23   24
[2,]   25   26   27   28
[3,]   29   30   31   32
[4,]   33   34   35   36
[5,]   37   38   39   40
\end{code}
%----------------------------------------------------------------------------------------------------------------------%

%----------------------------------------------------------------------------------------------------------------------%
\question{Find the range of values, max value, min value, mean, median, variance, sum of a vector.}
\begin{code}
    {Program}{r}
a <- seq(from = 1, to = 100, by = 3)
range(a)
max(a)
min(a)
mean(a)
median(a)
var(a)
sum(a)
\end{code}
\begin{code}
    {Output}{text}
[1]   1 100
[1] 100
[1] 1
[1] 50.5
[1] 50.5
[1] 892.5
[1] 1717
\end{code}
\newpage
%----------------------------------------------------------------------------------------------------------------------%

%----------------------------------------------------------------------------------------------------------------------%
\question{Create a data frame from vectors.}
\begin{code}
    {Program}{r}
attended <- c(2, 4, 5, 6)
total <- rep(10, 4)

attendance <- data.frame(
  'Classes Attended' = attended, 'Total Classes' = total,
  row.names = c('BS106', 'A201', 'BS158', 'A104')
)

attendance
\end{code}
\begin{code}
    {Output}{text}
      Classes.Attended Total.Classes
BS106                2            10
A201                 4            10
BS158                5            10
A104                 6            10
\end{code}
%----------------------------------------------------------------------------------------------------------------------%

%----------------------------------------------------------------------------------------------------------------------%
\question{Perform the operation of adding a new row to a data frame.}
\begin{code}
    {Program}{r}
attended <- c(2, 4, 5, 6)
total <- rep(10, 4)

attendance <- data.frame(
  'Classes Attended' = attended, 'Total Classes' = total,
  row.names = c('BS106', 'A201', 'BS158', 'A104')
)

attendance <- rbind(attendance, 'ICT114'=c(2, 10))

attendance
\end{code}
\begin{code}
    {Output}{text}
BS106                 2            10
A201                  4            10
BS158                 5            10
A104                  6            10
ICT114                2            10
\end{code}
%----------------------------------------------------------------------------------------------------------------------%
