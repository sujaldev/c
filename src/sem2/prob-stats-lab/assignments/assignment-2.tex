\startassignment{Graphs \& Diagrams}
\graphicspath{{../figures/assignment2/}}

%----------------------------------------------------------------------------------------------------------------------%
\firstquestion{Perform operations like assignment, addition and multiplication on variables.}
\begin{code}
    {Program}{r}
a <- 42
b <- 24
a + b
a * c
\end{code}
\outputfigure{q1}
%----------------------------------------------------------------------------------------------------------------------%

%----------------------------------------------------------------------------------------------------------------------%
\firstquestion{Using R make a grouped bar plot.}
\begin{code}
    {Program}{r}
students <- c(
  "Alice", "Bob", "Eve"
)
subjects <- c(
  "Physics", "Chemistry", "Mathematics"
)
marks <- c(
  2, 5, 1,
  4, 1, 2,
  8, 7, 2
)

results <- matrix(marks, ncol = 3, nrow = 3)
rownames(results) <- subjects
colnames(results) <- students

barplot(results, beside = TRUE)
\end{code}
\newpage
\pagestyle{fancy}
\outputfigure{q2}
%----------------------------------------------------------------------------------------------------------------------%

%----------------------------------------------------------------------------------------------------------------------%
\firstquestion{Using R make a grouped bar plot.}
\begin{code}
    {Program}{r}
students <- c("Alice", "Bob", "Eve")
subjects <- c("Physics", "Chemistry", "Mathematics")
marks <- c(2, 5, 1, 4, 1, 2, 8, 7, 2)

results <- matrix(marks, ncol = 3, nrow = 3)
rownames(results) <- subjects
colnames(results) <- students

barplot(results)
\end{code}
\outputfigure{q3}
%----------------------------------------------------------------------------------------------------------------------%

%----------------------------------------------------------------------------------------------------------------------%
\newpage
\firstquestion{Make a horizontal bar plot.}
\begin{code}
    {Program}{r}
students <- c("Alice", "Bob", "Eve")
subjects <- c("Physics", "Chemistry", "Mathematics")
marks <- c(2, 5, 1, 4, 1, 2, 8, 7, 2)

results <- matrix(marks, ncol = 3, nrow = 3)
rownames(results) <- subjects
colnames(results) <- students

barplot(results, horiz = TRUE, las = 1)
\end{code}
\outputfigure[0.5]{q4}
%----------------------------------------------------------------------------------------------------------------------%

%----------------------------------------------------------------------------------------------------------------------%
\firstquestion{Make a pie chart.}
\begin{code}
    {Program}{r}
pie(
  c(70, 20, 10, 10),
  labels = c("Sky", "Sunny Side", "Shaded Side", "Projection"),
  col = c("#5B8AE7", "#D2A353", "#583925", "#163F92")
)
\end{code}
\outputfigure[0.5]{q5}
%----------------------------------------------------------------------------------------------------------------------%

%----------------------------------------------------------------------------------------------------------------------%
\firstquestion{Make a pie chart showing atmospheric composition.}
\begin{code}
    {Program}{r}
pie(
  c(78, 21, 1, 1),
  labels = c("Nitrogen", "Oxygen", "Argon", "Carbon Dioxide"),
)
\end{code}
\outputfigure{q6}
%----------------------------------------------------------------------------------------------------------------------%

%----------------------------------------------------------------------------------------------------------------------%
\firstquestion{Make a histogram showing frequency of random numbers between 0 and 100.}
\begin{code}
    {Program}{r}
x <- runif(100)
hist(x)
\end{code}
\outputfigure{q7}
%----------------------------------------------------------------------------------------------------------------------%

%----------------------------------------------------------------------------------------------------------------------%
\newpage
\firstquestion{Generate a plot using random points.}
\begin{code}
    {Program}{r}
plot(
  runif(100),
  runif(100),
  xlab = "x-position",
  ylab = "y-position"
)
\end{code}
\outputfigure{q8}
%----------------------------------------------------------------------------------------------------------------------%

%----------------------------------------------------------------------------------------------------------------------%
\firstquestion{Plot the fibonacci series.}
\begin{code}
    {Program}{r}
plot(runif(100), type = "l")
\end{code}
\outputfigure{q9}
%----------------------------------------------------------------------------------------------------------------------%

%----------------------------------------------------------------------------------------------------------------------%
\newpage
\firstquestion{Plot the Fibonacci sequence.}
\begin{code}
    {Program}{r}
data <- c(0, 1, 1, 2, 3, 5, 8, 13, 21, 34, 55, 89, 144, 233, 377, 610)
plot(data, data, type = "b", xlab = "x", ylab = "y")
\end{code}
\outputfigure{q10}
%----------------------------------------------------------------------------------------------------------------------%
