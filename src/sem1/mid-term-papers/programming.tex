%! Author = Sujal Singh
%! Date = 11/1/23

% Preamble
\documentclass{ipu-papers}
\exam{Mid Term Examination (2023)}
\papercode{ICT-101 (I Sem)}
\subject{Programming for problem solving}
\duration{1:30 hrs}
\maxmarks{30}
\note{All questions are compulsory.}

% Packages
\usepackage{amsmath}
\usepackage{tabularx}
\usepackage[outputdir=../../build]{minted}

\usemintedstyle{bw}

% Document
\begin{document}
    \maketitle

    \begin{qsection}
        \question[$2\times{5} = 10$~Marks]{Answer the following questions.}
        \begin{subqsection}
            \question{What are preprocessor directives?}
            \question{What are the advantages of high level languages over machine language?}
            \question{Write a program using while loop to find sum of digits of a number entered by the user.}
            \question{What is output of printf(``\%d'');~?}
            \question{What is the difference between the priority and associativity in the operators?}
        \end{subqsection}
        %--------------------------------------------------------------------------------------------------------------%
        \question{}
        \begin{subqsection}
            \question[6 Marks]{WAP to multiply two matrices and draw its flowchart.}
            \question[4 Marks]{WAP to print the following pattern. User should be able to enter the number of rows to
            generate given pattern}\vspace*{-20pt}
            \begin{verbatim}
                                    ******
                                    *   *
                                    *  *
                                    * *
                                    **
                                    *
            \end{verbatim}
        \end{subqsection}
        %--------------------------------------------------------------------------------------------------------------%
        \question{}
        \begin{subqsection}
            \question[$2.5\times{2} = 5$~Marks]{Predict output of following code}\\
            \begin{tabular}{|p{0.46\textwidth}|p{0.46\textwidth}|}
                \hline
                i. & ii. \\
                \begin{minipage}{0.46\textwidth}
                    \begin{minted}{c}
#include<stdio.h>
int main()
{
int x=20, y=35;
x= y++ + x++;
y=++y + ++x;
printf("%d %d \n", x, y);
return 0;
}

                    \end{minted}
                \end{minipage}
                & \begin{minipage}{0.46\textwidth}
                      \begin{minted}{c}
#include<stdio.h>
int main()
{
int x=2, y=5;
y=2*y+x;
y=2*x+y;
printf("%d \n", x);
return 0;
}

                      \end{minted}
                \end{minipage} \\
                \hline
            \end{tabular}
            \newpage
            %--------------------------------------------------------------------------------------------------------------%
            \question[$2.5\times{2} = 5$~Marks]{Find the error in following code and correct the error}
            \begin{subsubqsection}
                \question{}\vspace*{-15pt}
                \begin{minted}{c}
#include <stdio.h>;
int main()
{
int integer1, integer2, sum;           /*declaration*/
printf("enter first integer\n")        /*prompt for first input*/
scanf("%d", integer1);                 /read integer value into integer1*/
printf("Enter second integer\n");      /*prompt for second input*/
scanf("%d", &integer2);                /*read integer value into integer2*/
sum = int1 + int2;                     /*add inputs and assign to sum*/
printf("Sum is %d\n", sum);            /*print sum*/
return 0;                              /*normal termination of program
}
                \end{minted}
                %--------------------------------------------------------------------------------------------------------------%
                \vspace*{20pt}
                \question{}\vspace*{-15pt}
                \begin{minted}{c}
main( )
{
  int size ;
  scanf ( "%d", &size );
  int arr[size] ;
  for ( i = 1 ; i <= size ; i++ )
  {
  scanf ( "%d", arr[i] ) ;
  printf ( "%d", arr[i] ) ;
  }
}
                \end{minted}
            \end{subsubqsection}
        \end{subqsection}
    \end{qsection}
\end{document}