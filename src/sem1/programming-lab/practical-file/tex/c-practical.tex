%! Author = Sujal Singh
%! Date = 10/18/23

% Preamble
%! suppress = FileNotFound
\documentclass[11pt]{ipu-c}
\usepackage[
    pdftitle={C Programming Lab Practical File},
    pdfsubject={C Programming Lab Practical File},
    pdfauthor={Sujal Singh},
    pdfdisplaydoctitle,
    hidelinks,
]{hyperref}

% Packages
\usepackage{amsmath}
\usepackage[indLines]{algpseudocodex}

% Document
\begin{document}
    \maketitle

    %------------------------------------------------------------------------------------------------------------------%

    \experiment{1}{Write a C program to print ``Hello, World!''}{Program to print ``Hello, World''.}

    \begin{tabularsection}{Algorithm}
        \begin{algorithmic}[1]
            \State Start
            \State{\textbf{print} ``Hello, World!''}
            \State Stop
        \end{algorithmic}
    \end{tabularsection}

    \begin{flowchart}
        \begin{tikzpicture}[node distance=2cm]
            \node (start) [startstop] {Start};
            \node (print) [io, below of=start] {print ``Hello, World!''};
            \node (stop) [startstop,below of=print] {Stop};

            \draw [arrow] (start) -- (print);
            \draw [arrow] (print) -- (stop);
        \end{tikzpicture}
    \end{flowchart}

    \begin{code}
        {Program}{c}
#include <stdio.h>

int main() {
    printf("Hello, World!");
    return 0;
}
    \end{code}
    \begin{code}
        {Input \& Output}{text}
Hello, World!
    \end{code}

    %------------------------------------------------------------------------------------------------------------------%
    %------------------------------------------------------------------------------------------------------------------%

    \experiment{2}{Write a C program to find the greatest number among three numbers provided by the user using if
    else.}{Program that finds the largest number among three numbers input by the user.}

    \begin{tabularsection}{Algorithm}
        \begin{algorithmic}[1]
            \State Start
            \State{\textbf{read} $a, b, c$}
            \If{$a \geq b ~\textbf{and}~ a \geq c$}
                \State $\text{largest} \gets a$

            \ElsIf
                    {$b \geq a ~\textbf{and}~ b \geq c$}
                \State $\text{largest} \gets b$
            \Else
                \State $\text{largest} \gets c$
            \EndIf
            \State \textbf{print} largest
            \State Stop
        \end{algorithmic}
    \end{tabularsection}

    \begin{flowchart}
        \begin{tikzpicture}[transform canvas={scale=0.7},node distance=2cm]
            \node (start) [startstop] {Start};
            \node (input) [io, below of=start] {read $a, b, c$};
            \node (checkA) [decision, below of=input,yshift=-40pt] {$a \geq b ~\textbf{and}~ a \geq c$};
            \node (setA) [process, right of=checkA,xshift=80pt] {$\text{largest} = a$};
            \node (checkB) [decision, below of=checkA,yshift=-80pt] {$b \geq a ~\textbf{and}~ b \geq c$};
            \node (setB) [process, left of=checkB,xshift=-80pt] {$\text{largest} = b$};
            \node (checkC) [process, below of=checkB,yshift=-40pt] {$\text{largest} = c$};
            \node (output) [io, below of=checkC] {print largest};
            \node (stop) [startstop,below of=output] {Stop};

            \draw [arrow] (start) -- (input);
            \draw [arrow] (input) -- (checkA);
            \draw [arrow] (checkA) -- node [anchor=south] {yes} (setA);
            \draw [arrow] (checkA) -- node [anchor=east] {no} (checkB);
            \draw [arrow] (checkB) -- node [anchor=south] {yes} (setB);
            \draw [arrow] (checkB) -- node [anchor=east] {no} (checkC);
            \draw [arrow] (checkC) -- (output);
            \draw [arrow] (output) -- (stop);
            \draw [arrow] (setA) |- (output);
            \draw [arrow] (setB) |- (output);
        \end{tikzpicture}
        \\[350pt]~
    \end{flowchart}

    \newpage
    \begin{code}
        {Program}{c}
#include <stdio.h>

int main() {
    int a, b, c, largest;

    printf("Enter first number: ");
    scanf("%d", &a);

    printf("Enter second number: ");
    scanf("%d", &b);

    printf("Enter third number: ");
    scanf("%d", &c);

    if (a >= b && a >= c) {
        largest = a;
    } else if (b >= a && b >= c) {
        largest = b;
    } else {
        largest = c;
    }

    printf("%d is the largest number.", largest);
    return 0;
}
    \end{code}
    \begin{code}
        {Input \& Output}{text}
Enter first number: 1
Enter second number: 2
Enter third number: 3
3 is the largest number.
    \end{code}

    %------------------------------------------------------------------------------------------------------------------%
    %------------------------------------------------------------------------------------------------------------------%

    \experiment{3}{Write a C program to find the sum of individual digits using while.}{Program that uses a while loop
    to find the sum of the individual digits in a number input by the user.}

    \begin{tabularsection}{Algorithm}
        \begin{algorithmic}[1]
            \State Start
            \State rem, sum \gets 0
            \State \textbf{read} num
            \While{num $\neq 0$}
                \State rem $\gets \text{num} \mod 10$
                \State sum $\gets \text{sum} + \text{rem}$
                \State num $\gets \frac{\text{num}}{10}$
            \EndWhile
            \State \textbf{print} sum
            \State Stop
        \end{algorithmic}
    \end{tabularsection}

    \begin{flowchart}
        \begin{tikzpicture}[transform canvas={scale=0.95}, node distance=2cm]
            \node (start) [startstop] {Start};
            \node (initBuffers) [process, below of=start] {rem, sum $= 0$};
            \node (input) [io, below of=initBuffers] {\textbf{read} num};
            \node (whileCondition) [decision, below of=input,yshift=-20pt] {num $\neq 0$};
            \node (s1) [process, right of=whileCondition,xshift=100pt] {rem $= \text{num}\mod{10}$};
            \node (s2) [process, below of=s1] {sum $= \text{sum} + \text{rem}$};
            \node (s3) [process, below of=s2] {num $= \frac{\text{num}}{10}$};
            \node (output) [io, left of=whileCondition,xshift=-100pt] {\textbf{print} sum};
            \node (stop) [startstop,below of=output] {Stop};

            \draw [arrow] (start) -- (initBuffers);
            \draw [arrow] (initBuffers) -- (input);
            \draw [arrow] (input) -- (whileCondition);
            \draw [arrow] (whileCondition) -- node[anchor=south] {yes} (s1);
            \draw [arrow] (whileCondition) -- node[anchor=south] {no} (output);
            \draw [arrow] (s1) -- (s2);
            \draw [arrow] (s2) -- (s3);
            \draw [arrow] (s3) -| (whileCondition);
            \draw [arrow] (output) -- (stop);
        \end{tikzpicture}
        \\[325pt]
    \end{flowchart}

    \newpage
    \begin{code}
        {Program}{c}
#include <stdio.h>

int main() {
    int num, rem, sum = 0;

    printf("Enter a number: ");
    scanf("%d", &num);

    while (num != 0) {
        rem = num % 10;
        sum += rem;
        num /= 10;
    }

    printf("Sum of digits = %d", sum);
    return 0;
}
    \end{code}
    \begin{code}
        {Input \& Output}{text}
Enter a number: 1234
Sum of digits = 10
    \end{code}

    %------------------------------------------------------------------------------------------------------------------%
\end{document}